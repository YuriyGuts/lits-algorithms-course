\documentclass[12pt,a4paper]{article}

\usepackage[unicode]{hyperref}
\pagenumbering{gobble}

% ----------------------------
% Enable multilingual content
% ----------------------------
\usepackage[T1, T2A]{fontenc}
\usepackage[utf8]{inputenc}
\usepackage[ukrainian, english]{babel}

% ---------------------
% Page layout settings
% ---------------------

% Set showframe to true to display layout guides.
\usepackage[showframe=false]{geometry}

% Enable this and use \layout in the document to include a special page with document layout metrics.
%\usepackage{layout}

% Set page margins
\geometry{margin=1in}

% ---------------------------
% Custom formatting for code
% ---------------------------

\usepackage{listings}
\usepackage{inconsolata}
\usepackage{upquote}


\lstdefinestyle{codePythonStyle}{
    language=Python,
    basewidth=0.53em,
    stepnumber=1,
    numbersep=8pt,
    tabsize=4,
    showspaces=false,
    showstringspaces=false
}

\lstset{
    basicstyle=\fontencoding{T1}\selectfont\ttfamily\small,
    style=codePythonStyle
}

% -------------------------------
% Custom formatting for headings
% -------------------------------
\usepackage{titlesec}

% Chapters
\titleformat
    {\chapter}                      % command
    [display]                       % shape
    {\large\bfseries}               % format
    {Модуль \thechapter}            % label
    {0.5ex}                         % sep
    {
        \rule{\textwidth}{1.5pt}
        \vspace{0.5ex}
        \centering
        \mdseries
        \itshape
    }                               % before-code
    [
        \vspace{-1.5ex}
        \rule{\textwidth}{0.3pt}
    ]                               % after-code

% Sections
\titleformat
    {\section}                      % command
    {\normalsize\bfseries}          % format
    {\thesection }                  % label
    {1em}                           % sep
    {}                              % before-code
    []                              % after-code

% ----------------
% List formatting
% ----------------
\usepackage{paralist}
    \let\itemize\compactitem
    \let\enditemize\endcompactitem
    \let\enumerate\compactenum
    \let\endenumerate\endcompactenum
    \let\description\compactdesc
    \let\enddescription\endcompactdesc
    \pltopsep=\medskipamount
    \plitemsep=1pt
    \plparsep=1pt

% ---------------------------------
% Custom formatting for paragraphs
% ---------------------------------
\setlength{\parskip}{0.5em}
\setlength{\parindent}{0em}

\newenvironment{pagebottomtext}{\begin{center}\par\vspace*{\fill}}{\clearpage\end{center}}




\begin{document}

\section*{Початок роботи}

Ласкаво просимо до курсу «Алгоритми та структури даних»!

Тут ви знайдете вказівки, як налаштувати інструменти для роботи.


\subsection*{Спілкування}

Для всіх учасників курсу існує груповий чат в Slack.
Ви отримаєте запрошення на вашу поштову адресу.
Прийміть його та налаштуйте свій особистий акаунт.

\begin{enumerate}
    \item Переконайтеся, що ви приєдналися до каналів |#general|, |#homework|, |#random|.
    \item Привітайтеся в каналі |#general|.
    \item Дуже рекомендується поставити своє фото в якості аватарки --- так іншим студентам буде легше вас впізнавати надалі. Також, заповніть свій профіль, вказавши реальне ім’я.
    \item Можете користуватися веб-версією чату або встановити \href{https://slack.com/downloads}{клієнт для комп’ютера чи смартфона}.
\end{enumerate}


\subsection*{Домашні завдання}

Для домашніх завдань працює автоматична система перевірки.
Щоб здавати домашні роботи, вам потрібно створити власний репозиторій на \href{https://github.com}{\color{blue}GitHub}.
Можна назвати його, наприклад, «|lits-ads-003|».

Після того, як створите репозиторій, надішліть його URL викладачу за допомогою приватного повідомлення у Slack.
Наприклад, |https://github.com/Petro/lits-ads-003|.

Також повідомте, якою мовою (мовами) ви збираєтеся вирішувати домашні завдання.
Ви зможете змінити свої вподобання в будь-який момент пізніше.


\subsection*{Конспекти, завдання, супровідні матеріали}

Матеріали курсу регулярно публікуються на спільну папку в Google Drive.

Ви повинні отримати поштою запрошення на перегляд цієї папки.

Додайте її до свого диску, щоб отримувати оновлення автоматично.


\end{document}