\documentclass[12pt,a4paper]{report}

% ----------------------------
% Enable multilingual content
% ----------------------------
\usepackage[T1, T2A]{fontenc}
\usepackage[utf8]{inputenc}
\usepackage[ukrainian, english]{babel}

% ---------------------
% Page layout settings
% ---------------------

% Set showframe to true to display layout guides.
\usepackage[showframe=false]{geometry}

% Enable this and use \layout in the document to include a special page with document layout metrics.
%\usepackage{layout}

% Set page margins
\geometry{margin=1in}

% ---------------------------
% Custom formatting for code
% ---------------------------

\usepackage{listings}
\usepackage{inconsolata}
\usepackage{upquote}


\lstdefinestyle{codePythonStyle}{
    language=Python,
    basewidth=0.53em,
    stepnumber=1,
    numbersep=8pt,
    tabsize=4,
    showspaces=false,
    showstringspaces=false
}

\lstset{
    basicstyle=\fontencoding{T1}\selectfont\ttfamily\small,
    style=codePythonStyle
}

% -------------------------------
% Custom formatting for headings
% -------------------------------
\usepackage{titlesec}

% Chapters
\titleformat
    {\chapter}                      % command
    [display]                       % shape
    {\large\bfseries}               % format
    {Модуль \thechapter}            % label
    {0.5ex}                         % sep
    {
        \rule{\textwidth}{1.5pt}
        \vspace{0.5ex}
        \centering
        \mdseries
        \itshape
    }                               % before-code
    [
        \vspace{-1.5ex}
        \rule{\textwidth}{0.3pt}
    ]                               % after-code

% Sections
\titleformat
    {\section}                      % command
    {\normalsize\bfseries}          % format
    {\thesection }                  % label
    {1em}                           % sep
    {}                              % before-code
    []                              % after-code

% ----------------
% List formatting
% ----------------
\usepackage{paralist}
    \let\itemize\compactitem
    \let\enditemize\endcompactitem
    \let\enumerate\compactenum
    \let\endenumerate\endcompactenum
    \let\description\compactdesc
    \let\enddescription\endcompactdesc
    \pltopsep=\medskipamount
    \plitemsep=1pt
    \plparsep=1pt

% ---------------------------------
% Custom formatting for paragraphs
% ---------------------------------
\setlength{\parskip}{0.5em}
\setlength{\parindent}{0em}

\newenvironment{pagebottomtext}{\begin{center}\par\vspace*{\fill}}{\clearpage\end{center}}




% TODO: Remove this after merging the modules into a single master file.
\setcounter{chapter}{1}


\begin{document}
%\layout
\chapter{Сортування.}



\section{Задача сортування.}

Часто сортування --- це найперша тема, яку вивчають студенти на дисциплінах «Основи програмування» чи «Алгоритми та структури даних».
Дуже ймовірно, що ви уже реалізовували самостійно деякі алгоритми сортування.
Однак, при цьому ваша улюблена мова програмування уже надає готові функції сортування у своїх системних бібліотеках.
Чому ж вартує вивчати алгоритми сортування та яку користь вони можуть принести?

\begin{itemize}
    \item Сортування --- одна з найпоширеніших задач, що виконується на комп’ютерах щоденно.
        Книга «Мистецтво програмування» Дональда Кнута наводить статистику одного дослідження 90-х років, яке показало, що чверть процесорного часу на мейнфреймах затрачалася на сортування даних.
        Відповідно, прискорення алгоритмів сортування суттєво впливає на продуктивність практично кожного комп’ютера в світі.
    \item Сортування --- це алгоритм, який лежить в основі багатьох складніших алгоритмів, або дозволяє їх відчутно прискорювати.
    \item Алгоритми сортування є хорошим джерелом алгоритмічних ідей.
        Так, парадигма «розділяй і володарюй» а також парадигма рандомізації лежать в основі двох швидких і популярних алгоритмів сортування --- \emph{MergeSort} і \emph{QuickSort}.
\end{itemize}


\subsection{Опис задачі.}

\begin{enumerate}
    \item Вхідні дані:
        \begin{itemize}
            \item Масив \(A\) довжиною \(N\) елементів.
            \item Функція \lstinline{compare(a, b)}, яка може порівняти два елементи \lstinline{a} та \lstinline{b}, і повідомити, який з цих елементів менший.

                \emph{Примітка 1.} На практиці, для простоти коду ця функція може повертати \lstinline{True}, якщо перший елемент менший за другий, і \lstinline{False} в іншому випадку.

                \lstinputlisting{code/compare_function_natural.py}

                \emph{Примітка 2.} Часто ця функція не фігурує в коді окремо, коли ми сортуємо елементи в натуральному порядку --- наприклад, числа за абсолютною величиною, або ж стрічки за алфавітом.
                Проте її корисно виділити для випадку, коли критерій сортування складніший --- наприклад, сортування чисел за сумою цифр, стрічок за довжиною тощо.

                \lstinputlisting{code/compare_function_complex.py}
        \end{itemize}
    \item Вихідні дані: вхідний масив, проте зі зміненим порядком елементів --- \(A'\). Елементи повинні бути розташованими таким чином, що будь-який елемент з меншим індексом є не більшим за будь-який елемент з більшим індексом.
\end{enumerate}



\end{document}