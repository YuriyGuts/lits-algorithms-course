\documentclass[12pt,a4paper]{article}

\usepackage{tikz}

% ----------------------------
% Enable multilingual content
% ----------------------------
\usepackage[T1, T2A]{fontenc}
\usepackage[utf8]{inputenc}
\usepackage[ukrainian, english]{babel}

\usepackage{bbding}
\usepackage{ifthen}
\usepackage{pgffor}

\newcommand{\fivestarrating}[1]{%
    \ifthenelse{\not\equal{#1}{0}}{\foreach \n in {1,...,#1}{\FiveStar}}{}%
    \ifthenelse{\not\equal{#1}{5}}{\foreach \n in {\numexpr(#1+1),...,5}{\FiveStarOpen}}{}%
}%

% ---------------------
% Page layout settings
% ---------------------

% Set showframe to true to display layout guides.
\usepackage[showframe=false]{geometry}

% Enable this and use \layout in the document to include a special page with document layout metrics.
%\usepackage{layout}

% Set page margins
\geometry{margin=1in}

% ---------------------------
% Custom formatting for code
% ---------------------------

\usepackage{listings}
\usepackage{inconsolata}
\usepackage{upquote}


\lstdefinestyle{codePythonStyle}{
    language=Python,
    basewidth=0.53em,
    stepnumber=1,
    numbersep=8pt,
    tabsize=4,
    showspaces=false,
    showstringspaces=false
}

\lstset{
    basicstyle=\fontencoding{T1}\selectfont\ttfamily\small,
    style=codePythonStyle
}

% -------------------------------
% Custom formatting for headings
% -------------------------------
\usepackage{titlesec}

% Chapters
\titleformat
    {\chapter}                      % command
    [display]                       % shape
    {\large\bfseries}               % format
    {Модуль \thechapter}            % label
    {0.5ex}                         % sep
    {
        \rule{\textwidth}{1.5pt}
        \vspace{0.5ex}
        \centering
        \mdseries
        \itshape
    }                               % before-code
    [
        \vspace{-1.5ex}
        \rule{\textwidth}{0.3pt}
    ]                               % after-code

% Sections
\titleformat
    {\section}                      % command
    {\normalsize\bfseries}          % format
    {\thesection }                  % label
    {1em}                           % sep
    {}                              % before-code
    []                              % after-code

% ----------------
% List formatting
% ----------------
\usepackage{paralist}
    \let\itemize\compactitem
    \let\enditemize\endcompactitem
    \let\enumerate\compactenum
    \let\endenumerate\endcompactenum
    \let\description\compactdesc
    \let\enddescription\endcompactdesc
    \pltopsep=\medskipamount
    \plitemsep=1pt
    \plparsep=1pt

% ---------------------------------
% Custom formatting for paragraphs
% ---------------------------------
\setlength{\parskip}{0.5em}
\setlength{\parindent}{0em}

\newenvironment{pagebottomtext}{\begin{center}\par\vspace*{\fill}}{\clearpage\end{center}}


% ---------------------------------------------
% Custom formatting for input/output examples
% ---------------------------------------------

\usepackage{xcolor}


\titleformat
    {\subsubsection}                % command
    [display]                       % shape
    {\normalsize\mdseries}          % format
    {}                              % label
    {0.5ex}                         % sep
    {
        \colorexample
    }                               % before-code

\newcommand{\colorexample}[1]{%
    \colorbox{gray!20}{%
        \parbox{%
            \dimexpr\textwidth-2\fboxsep
        }%
        {#1}
    }%
}



\begin{document}

\section*{Баг-трекер \hfill \fivestarrating{2}}


\subsection*{Код задачі: \code{BUGTRK}}

Після швидкого випуску нової версії своєї операційної системи, компанія NanoSoft почала отримувати від користувачів численні повідомлення про помилки.
Щоб слідкувати за прогресом виправлення помилок, команда розробників вирішила поставити велику квадратну дошку, і прикріпити до неї по листку для кожної помилки.
Проте, щоб дошка вмістилася в кімнаті, потрібно, щоб вона була не надто великою.

Всі \(N\) листків мають прямокутну форму і однаковий розмір \(W \times H\). Їх не можна повертати, а також накладати один на інший.

Знайдіть мінімальний розмір квадратної дошки, яка здатна вмістити всі листки.

Приклад мінімальної дошки, яка може вмістити 10 листків \(2 \times 3\):

\begin{center}
    \begin{tikzpicture}
        \draw[step=0.5cm,lightgray,very thin] (0,0) grid (4.5,4.5);
        \filldraw[fill=blue!20!white, draw=gray] (0,3) rectangle (1,4.5);
        \filldraw[fill=blue!20!white, draw=gray] (1.5,3) rectangle (2.5,4.5);
        \filldraw[fill=blue!20!white, draw=gray] (3.5,3) rectangle (4.5,4.5);
        \filldraw[fill=blue!20!white, draw=gray] (0,1.5) rectangle (1,3);
        \filldraw[fill=blue!20!white, draw=gray] (1,1.5) rectangle (2,3);
        \filldraw[fill=blue!20!white, draw=gray] (2.5,1.5) rectangle (3.5,3);
        \filldraw[fill=blue!20!white, draw=gray] (3.5,1.5) rectangle (4.5,3);
        \filldraw[fill=blue!20!white, draw=gray] (0,0) rectangle (1,1.5);
        \filldraw[fill=blue!20!white, draw=gray] (1.5,0) rectangle (2.5,1.5);
        \filldraw[fill=blue!20!white, draw=gray] (3,0) rectangle (4,1.5);
    \end{tikzpicture}
\end{center}


\subsection*{Вхідні дані}

Вхідний файл |bugtrk.in| складається з одного рядка.
Він містить три числа, розділені пробілом: \(N, W, H\) --- кількість листків, ширина та висота листка відповідно.

\begin{itemize}
    \item \(1 \leq N \leq 10^{12} \)
    \item \(1 \leq W \leq 10^{9} \)
    \item \(1 \leq H \leq 10^{9} \)
\end{itemize}


\subsection*{Вихідні дані}

Вихідний файл |bugtrk.out| повинен містити одне число --- мінімальна довжина сторони квадратної дошки.


\begin{pagebottomtext}
$\downarrow$ Див. приклади нижче $\downarrow$
\end{pagebottomtext}


\pagebreak


\subsubsection*{Приклад 1}

\textbf{\code{bugtrk.in}}

\begin{codeblock}
10 2 3
\end{codeblock}

\textbf{\code{bugtrk.out}}

\begin{codeblock}
9
\end{codeblock}


\subsubsection*{Приклад 2}

\textbf{\code{bugtrk.in}}

\begin{codeblock}
2 1000000000 999999999
\end{codeblock}

\textbf{\code{bugtrk.out}}

\begin{codeblock}
1999999998
\end{codeblock}


\subsubsection*{Приклад 3}

\textbf{\code{bugtrk.in}}

\begin{codeblock}
4 1 1
\end{codeblock}

\textbf{\code{bugtrk.out}}

\begin{codeblock}
2
\end{codeblock}


\end{document}