\documentclass[12pt,a4paper]{article}

% ----------------------------
% Enable multilingual content
% ----------------------------
\usepackage[T1, T2A]{fontenc}
\usepackage[utf8]{inputenc}
\usepackage[ukrainian, english]{babel}

\usepackage{bbding}
\usepackage{ifthen}
\usepackage{pgffor}

\newcommand{\fivestarrating}[1]{%
    \ifthenelse{\not\equal{#1}{0}}{\foreach \n in {1,...,#1}{\FiveStar}}{}%
    \ifthenelse{\not\equal{#1}{5}}{\foreach \n in {\numexpr(#1+1),...,5}{\FiveStarOpen}}{}%
}%

% ---------------------
% Page layout settings
% ---------------------

% Set showframe to true to display layout guides.
\usepackage[showframe=false]{geometry}

% Enable this and use \layout in the document to include a special page with document layout metrics.
%\usepackage{layout}

% Set page margins
\geometry{margin=1in}

% ---------------------------
% Custom formatting for code
% ---------------------------

\usepackage{listings}
\usepackage{inconsolata}
\usepackage{upquote}


\lstdefinestyle{codePythonStyle}{
    language=Python,
    basewidth=0.53em,
    stepnumber=1,
    numbersep=8pt,
    tabsize=4,
    showspaces=false,
    showstringspaces=false
}

\lstset{
    basicstyle=\fontencoding{T1}\selectfont\ttfamily\small,
    style=codePythonStyle
}

% -------------------------------
% Custom formatting for headings
% -------------------------------
\usepackage{titlesec}

% Chapters
\titleformat
    {\chapter}                      % command
    [display]                       % shape
    {\large\bfseries}               % format
    {Модуль \thechapter}            % label
    {0.5ex}                         % sep
    {
        \rule{\textwidth}{1.5pt}
        \vspace{0.5ex}
        \centering
        \mdseries
        \itshape
    }                               % before-code
    [
        \vspace{-1.5ex}
        \rule{\textwidth}{0.3pt}
    ]                               % after-code

% Sections
\titleformat
    {\section}                      % command
    {\normalsize\bfseries}          % format
    {\thesection }                  % label
    {1em}                           % sep
    {}                              % before-code
    []                              % after-code

% ----------------
% List formatting
% ----------------
\usepackage{paralist}
    \let\itemize\compactitem
    \let\enditemize\endcompactitem
    \let\enumerate\compactenum
    \let\endenumerate\endcompactenum
    \let\description\compactdesc
    \let\enddescription\endcompactdesc
    \pltopsep=\medskipamount
    \plitemsep=1pt
    \plparsep=1pt

% ---------------------------------
% Custom formatting for paragraphs
% ---------------------------------
\setlength{\parskip}{0.5em}
\setlength{\parindent}{0em}

\newenvironment{pagebottomtext}{\begin{center}\par\vspace*{\fill}}{\clearpage\end{center}}




\begin{document}

\section*{Дисконтна програма \hfill \fivestarrating{2}}


\subsection*{Код задачі: \code{DISCNT}}

Магазин модного одягу розпочав сезонний розпродаж, і пропонує покупцям знижку на кожен 3-ій придбаний товар.
Ви визначилися, які товари вас цікавлять, і бажаєте скористатися цією пропозицією, щоб максимально зекономити кошти.
Лишилося лише придумати, в якому порядку ви купуватимете товари.

Знаючи ціну кожного товару, а також відсоток знижки, підрахуйте, яку мінімальну суму вам доведеться заплатити за всі товари.


\subsection*{Вхідні дані}

Вхідний файл |discnt.in| складається з двох рядків.

\begin{itemize}
    \item Перший рядок містить перелік цілих чисел від 0 до 1000 включно, розділених пробілом --- ціни окремих товарів. Загальна кількість товарів не перевищує 1~000~000.
    \item Другий рядок містить |discount| --- ціле число від 0 до 100 включно, яке означає відсоток знижки за кожен третій товар.
\end{itemize}


\subsection*{Вихідні дані}

Вихідний файл |discnt.out| повинен містити одне число --- мінімальна сума, яку потрібно витратити, щоб придбати всі товари. Число повинне завжди мати два дробові знаки, і округлюватися до другого знаку після крапки.


\pagebreak


\subsection*{Приклади}

\subsubsection*{Приклад 1}

\textbf{\code{discnt.in}}

\begin{codeblock}
50 20 30 17 100
10
\end{codeblock}

\textbf{\code{discnt.out}}

\begin{codeblock}
207.00
\end{codeblock}
\emph{Пояснення:} Можна придбати в такому порядку: \((20 + 50 + 100\cdot0.9 + 17 + 30) = 207\)


\subsubsection*{Приклад 2}

\textbf{\code{discnt.in}}

\begin{codeblock}
1 2 3 4 5 6 7
100
\end{codeblock}

\textbf{\code{discnt.out}}

\begin{codeblock}
15.00
\end{codeblock}


\subsubsection*{Приклад 3}

\textbf{\code{discnt.in}}

\begin{codeblock}
1 1 1
33
\end{codeblock}

\textbf{\code{discnt.out}}

\begin{codeblock}
2.67
\end{codeblock}


\end{document}