\documentclass[12pt,a4paper]{article}

% ----------------------------
% Enable multilingual content
% ----------------------------
\usepackage[T1, T2A]{fontenc}
\usepackage[utf8]{inputenc}
\usepackage[ukrainian, english]{babel}

% ---------------------
% Page layout settings
% ---------------------

% Set showframe to true to display layout guides.
\usepackage[showframe=false]{geometry}

% Enable this and use \layout in the document to include a special page with document layout metrics.
%\usepackage{layout}

% Set page margins
\geometry{margin=1in}

% ---------------------------
% Custom formatting for code
% ---------------------------

\usepackage{listings}
\usepackage{inconsolata}
\usepackage{upquote}


\lstdefinestyle{codePythonStyle}{
    language=Python,
    basewidth=0.53em,
    stepnumber=1,
    numbersep=8pt,
    tabsize=4,
    showspaces=false,
    showstringspaces=false
}

\lstset{
    basicstyle=\fontencoding{T1}\selectfont\ttfamily\small,
    style=codePythonStyle
}

% -------------------------------
% Custom formatting for headings
% -------------------------------
\usepackage{titlesec}

% Chapters
\titleformat
    {\chapter}                      % command
    [display]                       % shape
    {\large\bfseries}               % format
    {Модуль \thechapter}            % label
    {0.5ex}                         % sep
    {
        \rule{\textwidth}{1.5pt}
        \vspace{0.5ex}
        \centering
        \mdseries
        \itshape
    }                               % before-code
    [
        \vspace{-1.5ex}
        \rule{\textwidth}{0.3pt}
    ]                               % after-code

% Sections
\titleformat
    {\section}                      % command
    {\normalsize\bfseries}          % format
    {\thesection }                  % label
    {1em}                           % sep
    {}                              % before-code
    []                              % after-code

% ----------------
% List formatting
% ----------------
\usepackage{paralist}
    \let\itemize\compactitem
    \let\enditemize\endcompactitem
    \let\enumerate\compactenum
    \let\endenumerate\endcompactenum
    \let\description\compactdesc
    \let\enddescription\endcompactdesc
    \pltopsep=\medskipamount
    \plitemsep=1pt
    \plparsep=1pt

% ---------------------------------
% Custom formatting for paragraphs
% ---------------------------------
\setlength{\parskip}{0.5em}
\setlength{\parindent}{0em}

\newenvironment{pagebottomtext}{\begin{center}\par\vspace*{\fill}}{\clearpage\end{center}}



\usepackage{hyperref}


\begin{document}

\section*{Як вирішувати задачі}


\subsection*{Формат програми}

Розв'язком кожної задачі має бути консольна програма, яка читає дані з вказаного в умові файлу, і записує результати у вказаний в умові файл.

\begin{enumerate}
    \item Не потрібно робити ввід/вивід з клавіатури чи графічний інтерфейс.
    \item Імена файлів вводу/виводу можна «хардкодити» в програмі.

        \vspace{0.5em}
        \emph{Корисно:} Щоб легше було відлагоджувати програму і тестувати її на різних наборах даних, ви можете (на власний розсуд) передбачити додаткову функцію: якщо програмі були передані імена файлів в командному рядку, тоді використовувати їх, інакше --- імена за замовчуванням.

        \vspace{0.5em}
        Приклад:
        \begin{itemize}
            \item |"movrat.exe"| --- читатиме дані з |movrat.in| та виводитиме результати в |movrat.out|.
            \item |"movrat.exe case1.in case1.out"| --- читатиме дані з |case1.in| та виводитиме результати в |case1.out|.
        \end{itemize}

    \item Коректність вхідних даних не потрібно перевіряти.
    \item Формат чисел для вхідних/вихідних даних:
        \begin{itemize}
            \item Розділювачів тисяч немає. Наприклад, 5000, 1000000, 123456789000.
            \item Дробовий знак --- крапка. Наприклад, 0.333, 12.75, 9.0.
        \end{itemize}
\end{enumerate}


\subsection*{Як здавати рішення}

\begin{enumerate}
    \item Створіть акаунт на \href{https://github.com}{GitHub}.
    \item Створіть репозиторій з довільним іменем, наприклад, |ads-001|.
    \item Для кожної задачі створюйте окрему папку, наприклад |/ads-001/movrat|, |/ads-001/discnt|.
    \item В кожній папці задачі зберігайте вихідний код програми.

    Наприклад, |/ads-001/movrat/movrat.py|. Файлів може бути більше ніж один, проте бажано, щоб головна програма називалася так само, як задача. Скомпільованих бінарних файлів зберігати не потрібно (використовуйте файл |.gitignore|: детальніше --- \href{http://git-scm.com/docs/gitignore}{документація}, \href{https://github.com/github/gitignore}{зразки на GitHub})
\end{enumerate}


\end{document}