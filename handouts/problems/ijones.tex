\documentclass[12pt,a4paper]{article}

\usepackage{tikz}
\usepackage[unicode]{hyperref}

\usetikzlibrary{arrows.meta}
\usetikzlibrary{shapes.misc}

% ----------------------------
% Enable multilingual content
% ----------------------------
\usepackage[T1, T2A]{fontenc}
\usepackage[utf8]{inputenc}
\usepackage[ukrainian, english]{babel}

\usepackage{bbding}
\usepackage{ifthen}
\usepackage{pgffor}

\newcommand{\fivestarrating}[1]{%
    \ifthenelse{\not\equal{#1}{0}}{\foreach \n in {1,...,#1}{\FiveStar}}{}%
    \ifthenelse{\not\equal{#1}{5}}{\foreach \n in {\numexpr(#1+1),...,5}{\FiveStarOpen}}{}%
}%

% ---------------------
% Page layout settings
% ---------------------

% Set showframe to true to display layout guides.
\usepackage[showframe=false]{geometry}

% Enable this and use \layout in the document to include a special page with document layout metrics.
%\usepackage{layout}

% Set page margins
\geometry{margin=1in}

% ---------------------------
% Custom formatting for code
% ---------------------------

\usepackage{listings}
\usepackage{inconsolata}
\usepackage{upquote}


\lstdefinestyle{codePythonStyle}{
    language=Python,
    basewidth=0.53em,
    stepnumber=1,
    numbersep=8pt,
    tabsize=4,
    showspaces=false,
    showstringspaces=false
}

\lstset{
    basicstyle=\fontencoding{T1}\selectfont\ttfamily\small,
    style=codePythonStyle
}

% -------------------------------
% Custom formatting for headings
% -------------------------------
\usepackage{titlesec}

% Chapters
\titleformat
    {\chapter}                      % command
    [display]                       % shape
    {\large\bfseries}               % format
    {Модуль \thechapter}            % label
    {0.5ex}                         % sep
    {
        \rule{\textwidth}{1.5pt}
        \vspace{0.5ex}
        \centering
        \mdseries
        \itshape
    }                               % before-code
    [
        \vspace{-1.5ex}
        \rule{\textwidth}{0.3pt}
    ]                               % after-code

% Sections
\titleformat
    {\section}                      % command
    {\normalsize\bfseries}          % format
    {\thesection }                  % label
    {1em}                           % sep
    {}                              % before-code
    []                              % after-code

% ----------------
% List formatting
% ----------------
\usepackage{paralist}
    \let\itemize\compactitem
    \let\enditemize\endcompactitem
    \let\enumerate\compactenum
    \let\endenumerate\endcompactenum
    \let\description\compactdesc
    \let\enddescription\endcompactdesc
    \pltopsep=\medskipamount
    \plitemsep=1pt
    \plparsep=1pt

% ---------------------------------
% Custom formatting for paragraphs
% ---------------------------------
\setlength{\parskip}{0.5em}
\setlength{\parindent}{0em}

\newenvironment{pagebottomtext}{\begin{center}\par\vspace*{\fill}}{\clearpage\end{center}}


% ---------------------------------------------
% Custom formatting for input/output examples
% ---------------------------------------------

\usepackage{xcolor}


\titleformat
    {\subsubsection}                % command
    [display]                       % shape
    {\normalsize\mdseries}          % format
    {}                              % label
    {0.5ex}                         % sep
    {
        \colorexample
    }                               % before-code

\newcommand{\colorexample}[1]{%
    \colorbox{gray!20}{%
        \parbox{%
            \dimexpr\textwidth-2\fboxsep
        }%
        {#1}
    }%
}


\tikzset{
    basic/.style = {text width=5pt, align=center}
}


\begin{document}

\section*{Індіана Джонс і останній прямокутний обхід \hfill \fivestarrating{4}}


\subsection*{Код задачі: \code{IJONES}}

В пошуках Святого Грааля Індіана Джонс зіткнувся з небезпечним випробуванням.
Йому потрібно пройти крізь прямокутний коридор, який складається з крихких плит (пригадайте \href{https://www.youtube.com/watch?v=arMXzgiZsJQ}{\color{blue}сцену з фільму} «Індіана Джонс і останній хрестовий похід»).
На кожній плиті написана одна літера:

\begin{center}
    \begin{tikzpicture}
        \begin{scope}[every node/.style={basic}]
            \draw[step=1cm,gray,very thin] (0,0) grid (3,3);
            \node at (0.5,2.5) (n11) {|a|}; \node at (1.5,2.5) (n21) {|a|}; \node [black!50!green] at (2.5,2.5) (n31) {|a|};
            \node at (0.5,1.5) (n12) {|c|}; \node at (1.5,1.5) (n22) {|a|}; \node at (2.5,1.5) (n32) {|b|};
            \node at (0.5,0.5) (n13) {|d|}; \node at (1.5,0.5) (n23) {|e|}; \node [black!50!green] at (2.5,0.5) (n33) {|f|};
        \end{scope}
    \end{tikzpicture}
\end{center}

Можна починати з будь-якої плити в найлівішому стовпці. Виходом із коридору є верхня права та нижня права плити (для прикладу вище --- |a| та |f|).

Індіана спритний, і може переходити не лише на сусідню плиту, а й перестрибувати через кілька плит.
Проте, щоб не провалитися крізь підлогу, він повинен дотримуватися таких правил:
\begin{enumerate}
    \item Після кожного кроку Індіана повинен опинятися правіше, ніж був перед цим.

    \begin{center}
        \begin{tikzpicture}
            \begin{scope}[every node/.style={basic}]
                \draw[step=1cm,gray,very thin] (0,0) grid (3,3);
                \node at (0.5,2.5) (n11) {|a|}; \node at (1.5,2.5) (n21) {|a|}; \node at (2.5,2.5) (n31) {|a|};
                \node at (0.5,1.5) (n12) {|c|}; \node at (1.5,1.5) (n22) {|a|}; \node at (2.5,1.5) (n32) {|b|};
                \node at (0.5,0.5) (n13) {|d|}; \node at (1.5,0.5) (n23) {|e|}; \node at (2.5,0.5) (n33) {|f|};

                \path [draw,->,-{Latex[length=3mm]},thick,black!50!green] (n11) -- (n21);
                \path [draw,->,-{Latex[length=3mm]},thick,red] (n22) --node[cross out,draw,-]{} (n12);
                \path [draw,->,-{Latex[length=3mm]},thick,red] (n22) --node[cross out,draw,-]{} (n23);
                \path [draw,->,-{Latex[length=3mm]},thick,black!50!green] (n13) -- (n23);
            \end{scope}
        \end{tikzpicture}
    \end{center}

    \item Завжди можна переходити на одну плиту праворуч.

    \begin{center}
        \begin{tikzpicture}
            \begin{scope}[every node/.style={basic}]
                \draw[step=1cm,gray,very thin] (0,0) grid (3,3);
                \node at (0.5,2.5) (n11) {|a|}; \node at (1.5,2.5) (n21) {|a|}; \node at (2.5,2.5) (n31) {|a|};
                \node at (0.5,1.5) (n12) {|c|}; \node at (1.5,1.5) (n22) {|a|}; \node at (2.5,1.5) (n32) {|b|};
                \node at (0.5,0.5) (n13) {|d|}; \node at (1.5,0.5) (n23) {|e|}; \node at (2.5,0.5) (n33) {|f|};

                \path [draw,->,-{Latex[length=3mm]},thick,black!50!green] (n11) -- (n21);
                \path [draw,->,-{Latex[length=3mm]},thick,black!50!green] (n21) -- (n31);
                \path [draw,->,-{Latex[length=3mm]},thick,black!50!green] (n12) -- (n22);
                \path [draw,->,-{Latex[length=3mm]},thick,black!50!green] (n13) -- (n23);
                \path [draw,->,-{Latex[length=3mm]},thick,black!50!green] (n23) -- (n33);
                \path [draw,->,-{Latex[length=3mm]},thick,red] (n12) --node[cross out,draw,-]{} (n21);
                \path [draw,->,-{Latex[length=3mm]},thick,red] (n13) --node[cross out,draw,-]{} (n22);
            \end{scope}
        \end{tikzpicture}
    \end{center}

    \item Крім руху на одну плиту праворуч, можна перестрибувати, проте лише на ту саму літеру. Наприклад, з літери |a| можна перестрибнути на будь-яку іншу літеру |a| за умови, що ми цим ходом просунемося правіше.

    \begin{center}
        \begin{tikzpicture}
            \begin{scope}[every node/.style={basic}]
                \draw[step=1cm,gray,very thin] (0,0) grid (3,3);
                \node at (0.5,2.5) (n11) {|a|}; \node at (1.5,2.5) (n21) {|a|}; \node at (2.5,2.5) (n31) {|a|};
                \node at (0.5,1.5) (n12) {|c|}; \node at (1.5,1.5) (n22) {|a|}; \node at (2.5,1.5) (n32) {|b|};
                \node at (0.5,0.5) (n13) {|d|}; \node at (1.5,0.5) (n23) {|e|}; \node at (2.5,0.5) (n33) {|f|};

                \path [draw,->,-{Latex[length=3mm]},thick,black!50!green] (n11) -- (n21);
                \path [draw,->,-{Latex[length=3mm]},thick,black!50!green] (n11) -- (n22);
                \path [draw,->,-{Latex[length=3mm]},thick,black!50!green] (n11) edge[bend left] node {} (n31);
                \path [draw,->,-{Latex[length=3mm]},thick,black!50!green] (n21) -- (n31);
                \path [draw,->,-{Latex[length=3mm]},thick,black!50!green] (n22) -- (n31);
                \path [draw,->,-{Latex[length=3mm]},thick,red] (n11) edge[bend right] node [cross out,draw,-]{} (n33);
            \end{scope}
        \end{tikzpicture}
    \end{center}

\end{enumerate}

Для заданого коридору, підрахуйте, скільки всього існує способів пройти його успішно.

\subsection*{Вхідні дані}

Вхідний файл |ijones.in| складається з \(H + 1\) рядків.
\begin{itemize}
    \item Перший рядок містить два числа \(W\) і \(H\), розділені пробілом: \(W\) --- ширина коридору, \(H\) --- висота коридору, \(1 \leq W, H \leq 2000\).
    \item Кожен з наступних \(H\) рядків містить слово довжиною \(W\) символів, яке складається з малих латинських літер від |a| до |z|.
\end{itemize}

Кодування файлу --- UTF-8 (without BOM).


\subsection*{Вихідні дані}

Вихідний файл |ijones.out| повинен містити одне ціле число --- кількість різних шляхів для виходу з коридору.


\pagebreak


\subsubsection*{Приклад 1}

\textbf{\code{ijones.in}}

\begin{codeblock}
3 3
aaa
cab
def
\end{codeblock}

\textbf{\code{ijones.out}}

\begin{codeblock}
5
\end{codeblock}

\emph{Пояснення:} Існує 3 варіанти обходу, якщо починати з літери |a|, і по одному варіанту, якщо починати з літери |c| або |d|.


\subsubsection*{Приклад 2}

\textbf{\code{ijones.in}}

\begin{codeblock}
10 1
abcdefaghi
\end{codeblock}

\textbf{\code{ijones.out}}

\begin{codeblock}
2
\end{codeblock}


\subsubsection*{Приклад 3}

\textbf{\code{ijones.in}}

\begin{codeblock}
7 6
aaaaaaa
aaaaaaa
aaaaaaa
aaaaaaa
aaaaaaa
aaaaaaa
\end{codeblock}

\textbf{\code{ijones.out}}

\begin{codeblock}
201684
\end{codeblock}


\end{document}
