\documentclass[12pt,a4paper]{article}

% ----------------------------
% Enable multilingual content
% ----------------------------
\usepackage[T1, T2A]{fontenc}
\usepackage[utf8]{inputenc}
\usepackage[ukrainian, english]{babel}

\usepackage{bbding}
\usepackage{ifthen}
\usepackage{pgffor}

\newcommand{\fivestarrating}[1]{%
    \ifthenelse{\not\equal{#1}{0}}{\foreach \n in {1,...,#1}{\FiveStar}}{}%
    \ifthenelse{\not\equal{#1}{5}}{\foreach \n in {\numexpr(#1+1),...,5}{\FiveStarOpen}}{}%
}%

% ---------------------
% Page layout settings
% ---------------------

% Set showframe to true to display layout guides.
\usepackage[showframe=false]{geometry}

% Enable this and use \layout in the document to include a special page with document layout metrics.
%\usepackage{layout}

% Set page margins
\geometry{margin=1in}

% ---------------------------
% Custom formatting for code
% ---------------------------

\usepackage{listings}
\usepackage{inconsolata}
\usepackage{upquote}


\lstdefinestyle{codePythonStyle}{
    language=Python,
    basewidth=0.53em,
    stepnumber=1,
    numbersep=8pt,
    tabsize=4,
    showspaces=false,
    showstringspaces=false
}

\lstset{
    basicstyle=\fontencoding{T1}\selectfont\ttfamily\small,
    style=codePythonStyle
}

% -------------------------------
% Custom formatting for headings
% -------------------------------
\usepackage{titlesec}

% Chapters
\titleformat
    {\chapter}                      % command
    [display]                       % shape
    {\large\bfseries}               % format
    {Модуль \thechapter}            % label
    {0.5ex}                         % sep
    {
        \rule{\textwidth}{1.5pt}
        \vspace{0.5ex}
        \centering
        \mdseries
        \itshape
    }                               % before-code
    [
        \vspace{-1.5ex}
        \rule{\textwidth}{0.3pt}
    ]                               % after-code

% Sections
\titleformat
    {\section}                      % command
    {\normalsize\bfseries}          % format
    {\thesection }                  % label
    {1em}                           % sep
    {}                              % before-code
    []                              % after-code

% ----------------
% List formatting
% ----------------
\usepackage{paralist}
    \let\itemize\compactitem
    \let\enditemize\endcompactitem
    \let\enumerate\compactenum
    \let\endenumerate\endcompactenum
    \let\description\compactdesc
    \let\enddescription\endcompactdesc
    \pltopsep=\medskipamount
    \plitemsep=1pt
    \plparsep=1pt

% ---------------------------------
% Custom formatting for paragraphs
% ---------------------------------
\setlength{\parskip}{0.5em}
\setlength{\parindent}{0em}

\newenvironment{pagebottomtext}{\begin{center}\par\vspace*{\fill}}{\clearpage\end{center}}


% ---------------------------------------------
% Custom formatting for input/output examples
% ---------------------------------------------

\usepackage{xcolor}


\titleformat
    {\subsubsection}                % command
    [display]                       % shape
    {\normalsize\mdseries}          % format
    {}                              % label
    {0.5ex}                         % sep
    {
        \colorexample
    }                               % before-code

\newcommand{\colorexample}[1]{%
    \colorbox{gray!20}{%
        \parbox{%
            \dimexpr\textwidth-2\fboxsep
        }%
        {#1}
    }%
}



\begin{document}

\section*{Довгий покер \hfill \fivestarrating{3}}


\subsection*{Код задачі: \code{LNGPOK}}

Ви граєте в альтернативний варіант покеру, де кожен гравець має в руках |N| карт, і метою гри є набрати якомога довшу групу послідовних карт.

Колода складається з карт, які мають числову величину.
Також, в колоді присутні джокери.
Якщо в руці гравця є джокери, він може присвоїти кожному будь-яку величину на власний розсуд.

Вам роздали карти.
Визначте довжину найдовшої послідовності карт, яку ви можете скласти.

\subsection*{Вхідні дані}

Вхідний файл |lngpok.in| складається з одного рядка.
Він містить перелік цілих чисел від 0 до 1000000 включно, розділених пробілом --- величини окремих карт в руці.
Загальна кількість карт в руці не перевищує 10000.

Джокери позначаються величиною 0.

«Перехід через верхню межу» не дозволяється --- [999999, 1000000, 1, 2] не вважається коректною послідовністю.


\subsection*{Вихідні дані}

Вихідний файл |lngpok.out| повинен містити одне число --- довжина найдовшої послідовної групи, яку можна скласти з виданих карт.


\begin{pagebottomtext}
$\downarrow$ Див. приклади нижче $\downarrow$
\end{pagebottomtext}


\pagebreak


\subsubsection*{Приклад 1}

\textbf{\code{lngpok.in}}

\begin{codeblock}
0 10 15 50 0 14 9 12 40
\end{codeblock}

\textbf{\code{lngpok.out}}

\begin{codeblock}
7
\end{codeblock}
\emph{Пояснення:} Можна замінити один джокер на 11, інший --- на 13, і скласти таку послідовність: 9, 10, 11, 12, 13, 14, 15.


\subsubsection*{Приклад 2}

\textbf{\code{lngpok.in}}

\begin{codeblock}
1 1 1 2 1 1 3
\end{codeblock}

\textbf{\code{lngpok.out}}

\begin{codeblock}
3
\end{codeblock}


\subsubsection*{Приклад 3}

\textbf{\code{lngpok.in}}

\begin{codeblock}
5 6 5 6 5 6 5 6 5 6 5 0 0
\end{codeblock}

\textbf{\code{lngpok.out}}

\begin{codeblock}
4
\end{codeblock}


\end{document}