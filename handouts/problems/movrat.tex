\documentclass[12pt,a4paper]{article}

% ----------------------------
% Enable multilingual content
% ----------------------------
\usepackage[T1, T2A]{fontenc}
\usepackage[utf8]{inputenc}
\usepackage[ukrainian, english]{babel}

% ---------------------
% Page layout settings
% ---------------------

% Set showframe to true to display layout guides.
\usepackage[showframe=false]{geometry}

% Enable this and use \layout in the document to include a special page with document layout metrics.
%\usepackage{layout}

% Set page margins
\geometry{margin=1in}

% ---------------------------
% Custom formatting for code
% ---------------------------

\usepackage{listings}
\usepackage{inconsolata}
\usepackage{upquote}


\lstdefinestyle{codePythonStyle}{
    language=Python,
    basewidth=0.53em,
    stepnumber=1,
    numbersep=8pt,
    tabsize=4,
    showspaces=false,
    showstringspaces=false
}

\lstset{
    basicstyle=\fontencoding{T1}\selectfont\ttfamily\small,
    style=codePythonStyle
}

% -------------------------------
% Custom formatting for headings
% -------------------------------
\usepackage{titlesec}

% Chapters
\titleformat
    {\chapter}                      % command
    [display]                       % shape
    {\large\bfseries}               % format
    {Модуль \thechapter}            % label
    {0.5ex}                         % sep
    {
        \rule{\textwidth}{1.5pt}
        \vspace{0.5ex}
        \centering
        \mdseries
        \itshape
    }                               % before-code
    [
        \vspace{-1.5ex}
        \rule{\textwidth}{0.3pt}
    ]                               % after-code

% Sections
\titleformat
    {\section}                      % command
    {\normalsize\bfseries}          % format
    {\thesection }                  % label
    {1em}                           % sep
    {}                              % before-code
    []                              % after-code

% ----------------
% List formatting
% ----------------
\usepackage{paralist}
    \let\itemize\compactitem
    \let\enditemize\endcompactitem
    \let\enumerate\compactenum
    \let\endenumerate\endcompactenum
    \let\description\compactdesc
    \let\enddescription\endcompactdesc
    \pltopsep=\medskipamount
    \plitemsep=1pt
    \plparsep=1pt

% ---------------------------------
% Custom formatting for paragraphs
% ---------------------------------
\setlength{\parskip}{0.5em}
\setlength{\parindent}{0em}

\newenvironment{pagebottomtext}{\begin{center}\par\vspace*{\fill}}{\clearpage\end{center}}




\begin{document}

\section*{Кінорейтинги}


\subsection*{Код задачі: \lstinline{MOVRAT}}

Веб-сайт для кінолюбителів дозволяє користувачам виставляти оцінки для кожного фільму. Проте, щоб зменшити вплив тролів та нечесного накручування рейтингу, сайт може відкидати кілька найнижчих та найвищих оцінок, і не враховувати їх при обчисленні рейтингу фільму.

Кожен користувач виставляє для фільму цілочисельну оцінку від 0 до 100. Після відкидання зайвих оцінок, рейтинг обчислюється як середнє арифметичне оцінок, що залишилися.

Ваше завдання --- маючи масив оцінок користувачів, а також квоту, скільки найнижчих і найвищих оцінок треба відкинути, підрахувати рейтинг фільму.


\subsection*{Вхідні дані}

Вхідний файл \lstinline[language=]{movrat.in} складається з 4 рядків.

\begin{itemize}
    \item Перший рядок містить \lstinline{N} --- кількість оцінок користувачів. \lstinline{N} --- натуральне число від 1 до 10 000 000.
    \item Наступний рядок містить \lstinline{N} цілих чисел від 0 до 100, розділених пробілом --- оцінки окремих користувачів.
    \item Третій рядок містить \lstinline{lowIgnoreCount} --- ціле число від 0 до 99, яке означатиме кількість найнижчих оцінок, які треба відкинути.
    \item Четвертий рядок містить \lstinline{highIgnoreCount} --- ціле число від 0 до 99, яке означатиме кількість найвищих оцінок, які треба відкинути.
    \item Сума \lstinline{lowIgnoreCount + highIgnoreCount} завжди буде меншою за \lstinline{N}.
\end{itemize}


\subsection*{Вихідні дані}

Вихідний файл \lstinline[language=]{movrat.out} повинен містити одне ціле число --- рейтинг фільму. Дробова частина повинна відкидатися.


\subsection*{Приклади}

\subsubsection*{Приклад 1}

\textbf{\lstinline[language=]{movrat.in}}

\begin{lstlisting}[language=]
5
40 50 22 17 0
0
0
\end{lstlisting}

\textbf{\lstinline[language=]{movrat.out}}

\begin{lstlisting}[language=]
25
\end{lstlisting}


\subsubsection*{Приклад 2}

\textbf{\lstinline[language=]{movrat.in}}

\begin{lstlisting}[language=]
5
40 50 22 17 0
1
2
\end{lstlisting}

\textbf{\lstinline[language=]{movrat.out}}

\begin{lstlisting}[language=]
19
\end{lstlisting}


\subsubsection*{Приклад 3}

\textbf{\lstinline[language=]{movrat.in}}

\begin{lstlisting}[language=]
3
0 100 0
1
1
\end{lstlisting}

\textbf{\lstinline[language=]{movrat.out}}
\begin{lstlisting}[language=]
0
\end{lstlisting}


\end{document}